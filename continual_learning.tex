\chapter{Continual Learning}
\section{Introduction}
\textbf{Continual Learning} is motivated by the fact that human and other organisms has the ability to adapt, accumulate and exploit knowledge. A common setting for continual learning is to learn a sequence of contents one by one and behave as if they were observed simultaneously \citep{wang2023comprehensive}. Each task learned throughout the life time can be new skills, new examples of old skills, different environments, etc. This attribute of continual learning makes it also referred to as \textbf{incremental learning} or \textbf{lifelong learning}. 

Unlike conventional pipeline, where joint training is applied, continual learning is characterized by learning from dynamic data distributions. A major challenge is known as \textbf{catastrophic forgetting}, where \textit{adaptation to a new distribution generally results in a largely reduced ability to capture the old ones}. This dilemma is a facet of the trade-off between \textbf{learning plasticity} and \textbf{memory stability}: an excess of the former interferes with the latter, and vice versa. A good continual learning algorithm should obtain a strong \textbf{generalizability} to accommodate distribution differences within and between tasks. 
